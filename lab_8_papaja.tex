\PassOptionsToPackage{unicode=true}{hyperref} % options for packages loaded elsewhere
\PassOptionsToPackage{hyphens}{url}
%
\documentclass[english,man]{apa6}
\usepackage{lmodern}
\usepackage{amssymb,amsmath}
\usepackage{ifxetex,ifluatex}
\usepackage{fixltx2e} % provides \textsubscript
\ifnum 0\ifxetex 1\fi\ifluatex 1\fi=0 % if pdftex
  \usepackage[T1]{fontenc}
  \usepackage[utf8]{inputenc}
  \usepackage{textcomp} % provides euro and other symbols
\else % if luatex or xelatex
  \usepackage{unicode-math}
  \defaultfontfeatures{Ligatures=TeX,Scale=MatchLowercase}
\fi
% use upquote if available, for straight quotes in verbatim environments
\IfFileExists{upquote.sty}{\usepackage{upquote}}{}
% use microtype if available
\IfFileExists{microtype.sty}{%
\usepackage[]{microtype}
\UseMicrotypeSet[protrusion]{basicmath} % disable protrusion for tt fonts
}{}
\IfFileExists{parskip.sty}{%
\usepackage{parskip}
}{% else
\setlength{\parindent}{0pt}
\setlength{\parskip}{6pt plus 2pt minus 1pt}
}
\usepackage{hyperref}
\hypersetup{
            pdftitle={Exploring Dwell Times for Dynamic Fractals},
            pdfkeywords={fractals, dwell times, growth and decay sequences},
            pdfborder={0 0 0},
            breaklinks=true}
\urlstyle{same}  % don't use monospace font for urls
\usepackage{graphicx,grffile}
\makeatletter
\def\maxwidth{\ifdim\Gin@nat@width>\linewidth\linewidth\else\Gin@nat@width\fi}
\def\maxheight{\ifdim\Gin@nat@height>\textheight\textheight\else\Gin@nat@height\fi}
\makeatother
% Scale images if necessary, so that they will not overflow the page
% margins by default, and it is still possible to overwrite the defaults
% using explicit options in \includegraphics[width, height, ...]{}
\setkeys{Gin}{width=\maxwidth,height=\maxheight,keepaspectratio}
\setlength{\emergencystretch}{3em}  % prevent overfull lines
\providecommand{\tightlist}{%
  \setlength{\itemsep}{0pt}\setlength{\parskip}{0pt}}
\setcounter{secnumdepth}{0}

% set default figure placement to htbp
\makeatletter
\def\fps@figure{htbp}
\makeatother

% Manuscript styling
\usepackage{upgreek}
\captionsetup{font=singlespacing,justification=justified}

% Table formatting
\usepackage{longtable}
\usepackage{lscape}
% \usepackage[counterclockwise]{rotating}   % Landscape page setup for large tables
\usepackage{multirow}		% Table styling
\usepackage{tabularx}		% Control Column width
\usepackage[flushleft]{threeparttable}	% Allows for three part tables with a specified notes section
\usepackage{threeparttablex}            % Lets threeparttable work with longtable

% Create new environments so endfloat can handle them
% \newenvironment{ltable}
%   {\begin{landscape}\begin{center}\begin{threeparttable}}
%   {\end{threeparttable}\end{center}\end{landscape}}
\newenvironment{lltable}{\begin{landscape}\begin{center}\begin{ThreePartTable}}{\end{ThreePartTable}\end{center}\end{landscape}}

% Enables adjusting longtable caption width to table width
% Solution found at http://golatex.de/longtable-mit-caption-so-breit-wie-die-tabelle-t15767.html
\makeatletter
\newcommand\LastLTentrywidth{1em}
\newlength\longtablewidth
\setlength{\longtablewidth}{1in}
\newcommand{\getlongtablewidth}{\begingroup \ifcsname LT@\roman{LT@tables}\endcsname \global\longtablewidth=0pt \renewcommand{\LT@entry}[2]{\global\advance\longtablewidth by ##2\relax\gdef\LastLTentrywidth{##2}}\@nameuse{LT@\roman{LT@tables}} \fi \endgroup}

% \setlength{\parindent}{0.5in}
% \setlength{\parskip}{0pt plus 0pt minus 0pt}

% \usepackage{etoolbox}
\makeatletter
\patchcmd{\HyOrg@maketitle}
  {\section{\normalfont\normalsize\abstractname}}
  {\section*{\normalfont\normalsize\abstractname}}
  {}{\typeout{Failed to patch abstract.}}
\makeatother
\shorttitle{Fractal Dwell Times}
\author{Scott Wallner\textsuperscript{1, 2}, Tess Sameshima\textsuperscript{1, 2}, \& Benjamin Chaloupka\textsuperscript{1, 2}}
\affiliation{
\vspace{0.5cm}
\textsuperscript{1} University of Oregon\\\textsuperscript{2} Institute of Oregon}
\authornote{Perception and Action Lab, Cognitivie Dynamics Lab, \& Brain and Memory Lab


}
\keywords{fractals, dwell times, growth and decay sequences}
\DeclareDelayedFloatFlavor{ThreePartTable}{table}
\DeclareDelayedFloatFlavor{lltable}{table}
\DeclareDelayedFloatFlavor*{longtable}{table}
\makeatletter
\renewcommand{\efloat@iwrite}[1]{\immediate\expandafter\protected@write\csname efloat@post#1\endcsname{}}
\makeatother
\usepackage{csquotes}
\ifnum 0\ifxetex 1\fi\ifluatex 1\fi=0 % if pdftex
  \usepackage[shorthands=off,main=english]{babel}
\else
  % load polyglossia as late as possible as it *could* call bidi if RTL lang (e.g. Hebrew or Arabic)
  \usepackage{polyglossia}
  \setmainlanguage[]{english}
\fi

\title{Exploring Dwell Times for Dynamic Fractals}

\date{}

\abstract{
Our primary research question is whether dwell times systematically differ between growth and decay sequences. We will examine this question both within and across subjects. Additionally, we are possibly interested in several exploratory analyses. For example, do specific fractal images elicit longer dwell times? Do specific levels of complexity elicit longer dwell times? Is there a dwell time pattern when fractal iterations are presented randomly? Does dwell time systematically decrease over time (if so, we may need to normalize dwell times to account for this)?
}

\begin{document}
\maketitle

Brazhe and Maksimov (2006) say stuff about fractals and such. Another group of researchers does too (Lowen \& Teich, 1993).

\hypertarget{methods}{%
\section{Methods}\label{methods}}

We report how we determined our sample size, all data exclusions (if any), all manipulations, and all measures in the study.

\hypertarget{participants}{%
\subsection{Participants}\label{participants}}

\hypertarget{material}{%
\subsection{Material}\label{material}}

\hypertarget{procedure}{%
\subsection{Procedure}\label{procedure}}

\hypertarget{data-analysis}{%
\subsection{Data analysis}\label{data-analysis}}

We used R (Version 3.6.3; R Core Team, 2020) and the R-packages \emph{dplyr} (Version 1.0.2; Wickham et al., 2020), \emph{forcats} (Version 0.5.0; Wickham, 2020a), \emph{ggplot2} (Version 3.3.2; Wickham, 2016), \emph{here} (Version 0.1; Müller, 2017), \emph{janitor} (Version 2.0.1; Firke, 2020), \emph{papaja} (Version 0.1.0.9942; Aust \& Barth, 2020), \emph{purrr} (Version 0.3.4; Henry \& Wickham, 2020), \emph{readr} (Version 1.4.0; Wickham \& Hester, 2020), \emph{rio} (Version 0.5.16; Chan, Chan, Leeper, \& Becker, 2018), \emph{stringr} (Version 1.4.0; Wickham, 2019), \emph{tibble} (Version 3.0.4; Müller \& Wickham, 2020), \emph{tidyr} (Version 1.1.2; Wickham, 2020b), and \emph{tidyverse} (Version 1.3.0; Wickham, Averick, et al., 2019) for all our analyses.

\hypertarget{results}{%
\section{Results}\label{results}}

\begin{tabular}{l|l|r|r|r|r}
\hline
sex & frl & math\_mean & math\_sd & rdg\_mean & rdg\_sd\\
\hline
boy & no & 492.85 & 46.34 & 441.46 & 32.32\\
\hline
boy & yes & 469.87 & 46.09 & 425.38 & 26.63\\
\hline
girl & no & 501.21 & 45.96 & 448.54 & 34.52\\
\hline
girl & yes & 477.51 & 46.30 & 430.80 & 27.42\\
\hline
\end{tabular}

Girls who paid for their lunches had the highest reading and math scores on average. In both boys and girls, students who get free lunch scored lower in both math and reading.

\includegraphics{lab_8_papaja_files/figure-latex/unnamed-chunk-2-1.pdf}

On average, students who paid for their meals had a higher overall total math score. It does not seem that teacher experience has a strong effect on total math score. Perhaps the difference between paid or free meals relates to the students' SES which has an effect on total math score.

\hypertarget{discussion}{%
\section{Discussion}\label{discussion}}

\newpage

\hypertarget{references}{%
\section{References}\label{references}}

\begingroup
\setlength{\parindent}{-0.5in}
\setlength{\leftskip}{0.5in}

\hypertarget{refs}{}
\leavevmode\hypertarget{ref-R-papaja}{}%
Aust, F., \& Barth, M. (2020). \emph{papaja: Create APA manuscripts with R Markdown}. Retrieved from \url{https://github.com/crsh/papaja}

\leavevmode\hypertarget{ref-brazhe2006self}{}%
Brazhe, A., \& Maksimov, G. (2006). Self-organized critical gating of ion channels: On the origin of long-term memory in dwell time series. \emph{Chaos: An Interdisciplinary Journal of Nonlinear Science}, \emph{16}(3), 033129.

\leavevmode\hypertarget{ref-R-rio}{}%
Chan, C.-h., Chan, G. C., Leeper, T. J., \& Becker, J. (2018). \emph{Rio: A swiss-army knife for data file i/o}.

\leavevmode\hypertarget{ref-R-janitor}{}%
Firke, S. (2020). \emph{Janitor: Simple tools for examining and cleaning dirty data}. Retrieved from \url{https://CRAN.R-project.org/package=janitor}

\leavevmode\hypertarget{ref-R-purrr}{}%
Henry, L., \& Wickham, H. (2020). \emph{Purrr: Functional programming tools}. Retrieved from \url{https://CRAN.R-project.org/package=purrr}

\leavevmode\hypertarget{ref-lowen1993fractal}{}%
Lowen, S. B., \& Teich, M. C. (1993). Fractal renewal processes generate 1/f noise. \emph{Physical Review E}, \emph{47}(2), 992.

\leavevmode\hypertarget{ref-R-here}{}%
Müller, K. (2017). \emph{Here: A simpler way to find your files}. Retrieved from \url{https://CRAN.R-project.org/package=here}

\leavevmode\hypertarget{ref-R-tibble}{}%
Müller, K., \& Wickham, H. (2020). \emph{Tibble: Simple data frames}. Retrieved from \url{https://CRAN.R-project.org/package=tibble}

\leavevmode\hypertarget{ref-R-base}{}%
R Core Team. (2020). \emph{R: A language and environment for statistical computing}. Vienna, Austria: R Foundation for Statistical Computing. Retrieved from \url{https://www.R-project.org/}

\leavevmode\hypertarget{ref-R-ggplot2}{}%
Wickham, H. (2016). \emph{Ggplot2: Elegant graphics for data analysis}. Springer-Verlag New York. Retrieved from \url{https://ggplot2.tidyverse.org}

\leavevmode\hypertarget{ref-R-stringr}{}%
Wickham, H. (2019). \emph{Stringr: Simple, consistent wrappers for common string operations}. Retrieved from \url{https://CRAN.R-project.org/package=stringr}

\leavevmode\hypertarget{ref-R-forcats}{}%
Wickham, H. (2020a). \emph{Forcats: Tools for working with categorical variables (factors)}. Retrieved from \url{https://CRAN.R-project.org/package=forcats}

\leavevmode\hypertarget{ref-R-tidyr}{}%
Wickham, H. (2020b). \emph{Tidyr: Tidy messy data}. Retrieved from \url{https://CRAN.R-project.org/package=tidyr}

\leavevmode\hypertarget{ref-R-tidyverse}{}%
Wickham, H., Averick, M., Bryan, J., Chang, W., McGowan, L. D., François, R., \ldots{} Yutani, H. (2019). Welcome to the tidyverse. \emph{Journal of Open Source Software}, \emph{4}(43), 1686. \url{https://doi.org/10.21105/joss.01686}

\leavevmode\hypertarget{ref-R-dplyr}{}%
Wickham, H., François, R., Henry, L., \& Müller, K. (2020). \emph{Dplyr: A grammar of data manipulation}. Retrieved from \url{https://CRAN.R-project.org/package=dplyr}

\leavevmode\hypertarget{ref-R-readr}{}%
Wickham, H., \& Hester, J. (2020). \emph{Readr: Read rectangular text data}. Retrieved from \url{https://CRAN.R-project.org/package=readr}

\endgroup

\end{document}
